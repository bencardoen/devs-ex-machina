\documentclass[]{article}

%opening
\title{Notes on DXExMachina performance.}
\author{}

\begin{document}

\maketitle
\section{Accompanying documents}
Benchmarks (on master) in separate pdf. Further benchmarks and profiling data in repository.

\section{Improvements since abstract}
\begin{itemize}
	\item Conservative handles cyclic simulations.
	\item Scheduler ported from Adevs + own heuristic.
	\item Memory pooling.
	\subitem Aligning messages on cache lines to avoid false sharing
	\subitem Thread local pools to avoid expensive cross-thread alloc/dealloc.
	\subitem Gvt required for conservative, based on existing null message exchange.
	\item Frequency of locking reduced, (sparse) usage of memory\_order\_relaxed. Where possible explicit	locking switched to atomic.
	\item A (large) nr of bugfixes and code optimizations.
\end{itemize}
\section{Summarized results benchmarks}
For the detailed results we refer to accompanying pdf and profiling call graphs.
\subsection{Devstone}
Faster due to directconnect. Optimistic suffers from transitions at same time points (frequency reverts higher), can gain with random ta.
\subsection{Phold}
Slightly faster than adevs. A reasonable amount of messaging can be done without slowing down simulator. Most of the runtime is spent in rng.
\subsection{Interconnect}
Slower than adevs due to high volume of messages. This creates a very cache unfriendly pattern, combined with cost of allocation on the heap for dxex. 
The completely cyclic dependencies of this model makes it very hard for our parallel implementations. Usage of pools reduces calls to new/malloc/sbrk, but can't eliminate it. Cache pattern is far harder to correct. Alignment of a Message on cache lines, and quadratic growing pools also help, but can't solve the problem.
\section{Open issues to improve}
\begin{itemize}
	\item Conservative lookahead caching (devstone -r), cpdevs is forced to ask useless lookahead values.
	\item Optimistic use pools in states.
	\item Directconnect : Yentl's faster algorithm. Not (yet) implement since it does not show in profiling graphs as a significant (<1\%) cost.
	\item Conservative : Exploit full dependency graph to calculate next event directly (unsure of benefits).
\end{itemize}
\section{Causes of slowdown (hardware)}
A short illustration of key differences in hardware events for both simulators.
\begin{itemize}
	\item Cache : Adevs : $<$ 1\% cache misses, dxex : $<$10\% cache misses
	\item Context switches : Adevs : 4e-8 switches/cycles, dxex 3e-8 (conservative)
	\item Branch prediction : Adevs : 4\%, dxex 0.08%
	\item Cpu stalls : Adevs : 0.2 stalled cycle/instruction, dxex 0.8 stalled cycle/instruction.
\end{itemize}
The switches are (in single core) caused by syscalls (malloc, sbrk, free, time), in parallel by kernel space locking (mutex, thread, (depending on implementation atomics)).


\end{document}
