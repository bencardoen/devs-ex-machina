\section{Setup}

After creating the models for your simulation, there are still a few parameters to take care of to get everything up and running.\\

The first part to take into consideration is the \textbf{allocator}, which decides on which simulation core to place which model. We have already included a simple allocator (called \textsl{SimpleAllocator}) that distributes models evenly, but for more complex behavior it is possible to create one tailored specifically for your simulation purposes by inheriting from the \textsl{Allocator} base class.\\

To get output from your simulation, you can select one or more of the included \textbf{tracers}. See \ref{sec:Tracing} for more details.\\
\\
The easiest way to get started is using the included \textsl{ControllerConfig}. This object will take care of a large part of the setup for you, though it is also possible to do everything manually.\\

\textsl{ControllerConfig} has a few options to configure, all of them with a certain default value.
\\\\
\begin{tabular}{lp{8cm}}
\textit{name} & The name of your simulation \newline
				By default this is \texttt{MySimulation}
\\\\
\textit{simType} & The type of DEVS you wish to run:

	\begin{tabular}{l|l}
	\textsf{Option} & \textsf{Meaning} \\
	\texttt{Controller::CLASSIC} & \emph{Classic DEVS} \\
	\texttt{Controller::DSDEVS} & \emph{Dynamic Structure DEVS} \\
	\texttt{Controller::PDEVS} & \emph{Parallel DEVS} \\
	\end{tabular} \newline
	
	The default is \emph{Classic DEVS}
\\\\
\end{tabular}
\\
\begin{tabular}{lp{8cm}}
\textit{coreAmount} & The amount of cores the simulation will run on \newline
						\textit{[This option is disregarded if you do not run PDEVS]} \newline
						By default this is 1
\\\\
\textit{saveInterval} & This value decides how often the simulation is saved and the tracing output dumped. A value of 3 would mean that this happens every 3 simulation cycles. \newline
						The default value is 5
\\\\
\textit{allocator} & By default this is the included \textsl{SimpleAllocator}
\\\\
\end{tabular}\\
\\
After you are done setting the right options, you can generate a \textbf{controller} by using \textsl{ControllerConfig}'s \textsl{createController} method. All further interactions with your simulation are done through this controller.\\
\\
As a short example, this configuration would give you a Classic DEVS named \texttt{foosim} that saves every 8 cycles:
\begin{verbatim}
ControllerConfig conf = ControllerConfig();
conf.name = "foosim";
conf.saveInterval = 8;

auto controller = conf.createController();
\end{verbatim}
The final step of the setup would be to add any models to the controller using the \textsl{addModel} method. Your simulation is now ready to run!

\section{Control}
You can choose to let the simulation end at a certain moment by adding a \textbf{termination time} or a \textbf{termination condition}.\\
The termination time can be set using the controller's \textsl{setTerminationTime} method, the termination condition by using \textsl{setTerminationCondition} and passing a custom \textsl{TerminationFunctor}.\\ %TODO add manual terminationfunctor?
\\
Finally, you can start the DEVS using the controller's \textsl{simulate} method.
%TODO Serializatie: save, load