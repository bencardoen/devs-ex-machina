\section{Setup}

After creating the models for your simulation, there are still a few parameters to take care of to get everything up and running.\\

The first part to take into consideration is the \textbf{allocator}, which decides on which simulation core to place which model. We have already included a simple allocator (called \textsl{SimpleAllocator}) that distributes models evenly, but for more complex behavior it is possible to create one tailored specifically to your simulation purposes by inheriting from the \textsl{Allocator} base class.\\

To get output from your simulation, you can select one or more of the included \textbf{tracers}. See \ref{sec:Tracing} for more details.\\
\\
The easiest way to get started is using the included \textsl{ControllerConfig}. This object will take care of a large part of the setup for you, though it is also possible to do everything manually.\\

\textsl{ControllerConfig} has a few options to configure, all of them with a certain default value.
\\\\
\begin{tabular}{lp{8cm}}
\textit{name} & The name of your simulation \newline
				By default this is \texttt{MySimulation}
\\\\
\textit{simType} & The type of DEVS you wish to run:

	\begin{tabular}{l|l}
	\textsf{Option} & \textsf{Meaning} \\
	\texttt{Controller::CLASSIC} & \emph{Classic DEVS} \\
	\texttt{Controller::DSDEVS} & \emph{Dynamic Structure DEVS} \\
	\texttt{Controller::PDEVS} & \emph{Parallel DEVS} \\
	\end{tabular} \newline
	
	The default is \emph{Classic DEVS}
\\\\
\end{tabular}
\\
\begin{tabular}{lp{8cm}}
\textit{simType} & The behavior of the Parallel DEVS:

	\begin{tabular}{l|l}
	\textsf{Option} & \textsf{Meaning} \\
	\texttt{OPTIMISTIC} & \emph{Optimistic PDEVS Simulation} \\
	\texttt{CONSERVATIVE} & \emph{Conservative PDEVS Simulation} \\
	\end{tabular} \newline
	\textit{[This option is disregarded if you do not run PDEVS]} \newline
	The default is \emph{OPTIMISTIC}
	
\\\\
\textit{coreAmount} & The amount of cores the simulation will run on \newline
						\textit{[This option is disregarded if you do not run PDEVS]} \newline
						By default this is 1
\\\\
\textit{saveInterval} & This value decides how often the simulation is saved and the tracing output dumped. A value of 3 would mean that this happens every 3 simulation cycles. \newline
\textit{[This option is disregarded if you do not run CLASSIC or DSDEVS]} \newline
						The default value is 5
\\\\
\textit{allocator} & By default this is the included \textsl{SimpleAllocator}
\\\\
\end{tabular}\\
\\
After you are done setting the right options, you can generate a \textbf{controller} by using \textsl{ControllerConfig}'s \textsl{createController} method. All further interactions with your simulation are done through this controller.\\
\\
As a short example, this configuration would give you a Classic DEVS named \texttt{foosim} that saves every 8 cycles:
\begin{verbatim}
ControllerConfig conf = ControllerConfig();
conf.name = "foosim";
conf.saveInterval = 8;

auto controller = conf.createController();
\end{verbatim}
The final step of the setup would be to add any models to the controller using the \textsl{addModel} method. Your simulation is now ready to run!

\section{Control}
You can choose to let the simulation end at a certain moment by adding a \textbf{termination time} or a \textbf{termination condition}.\\
The termination time can be set using the controller's \textsl{setTerminationTime} method, the termination condition by using \textsl{setTerminationCondition} and passing a custom \textsl{TerminationFunctor}.

\subsection{Events}
If you want to schedule certain events while the simulation is running, like pausing and saving, you can use the Controller's \textsl{addPauseEvent} and \textsl{addSaveEvent} methods.\\
\\
The \textsl{addPauseEvent} method requires the \emph{simulated time} at which the simulator has to pause and how long the pause should take (in seconds).\\
The \textsl{addSaveEvent} method requires the \emph{simulated time} at which the simulation has to be saved and the \textsl{prefix} of the saved file. For example: a save at the simulated time \textit{100} with the prefix \texttt{mySimSave} will result in a file \texttt{mySimSave\_100.devs}.

For both events, the boolean value \textit{true} can be added as an additional argument to make the event repeat itself until the simulation ends.

\subsubsection{Loading a previous simulation}
Loading a previous simulation from a savefile follows the same procedure as setting up a regular simulation for the most part.\\
First the simulation is configured (optionally by using the ControllerConfig object as described above) and then, instead of loading a new model, you can use the Controller's \textsl{load} method to load the data of your previous simulation.\\
\\
\textit{Important: to load a simulation you \textbf{do not} have to use the same configuration!}
\\
If you wish to increase the amount of cores of a loaded simulation, it is recommended to use an allocator which overrides the model's preferred location, since those will still be set to the cores of the previous simulation. The included \textsl{SimpleAllocator} can be used by setting its \texttt{allowUserOverride} flag to \texttt{false} in the constructor. Reducing the amount of cores is handled automatically.

\section{Starting the simulation}
Finally, you can start the DEVS using the controller's \textsl{simulate} method.
