The following are a few quick examples to illustrate the setup of a regular simulation.\\
Additionally, many working examples are provided in our Controller testing classes (\texttt{controllertest.cpp}). For each formalism a model was created which the user can use as a reference implementation.\\

\section{Classic DEVS}
The following example describes the manual setup of a crowd simulation in a zoo:
\begin{Verbatim}[fontsize=\small]
// Setting up all prerequisites
auto tracers = createObject<n_tracers::t_tracerset>();
std::unordered_map<std::size_t, t_coreptr> coreMap;
std::shared_ptr<Allocator> allocator = createObject<SimpleAllocator>(1);
std::shared_ptr<n_control::LocationTable> locTab = createObject<n_control::LocationTable>(1);

// Setting up the core
t_coreptr c = createObject<Core>();
coreMap[0] = c;

// Setup of the controller
Controller ctrl("zooSim", coreMap, allocator, locTab, tracers);
ctrl.setClassicDEVS();
ctrl.setTerminationTime(t_timestamp(360, 0));

// Creating and adding the main model
t_coupledmodelptr m1 = createObject<n_crowdSim::Zoo>("Planckendael");
ctrl.addModel(m1);

// Starting the simulation
ctrl.simulate();
\end{Verbatim}

In cases where you don't need control over all individual components, it's even quicker to opt for using \textsl{ControllerConfig} to set up everything:

\begin{Verbatim}[fontsize=\small]
// Creating a ControllerConfig object, setting all non-default parameters
ControllerConfig conf;
conf.name = "zooSim";

auto ctrl = conf.createController();
ctrl->setTerminationTime(t_timestamp(360, 0));

t_coupledmodelptr m1 = createObject<n_crowdSim::Zoo>("Planckendael");
ctrl->addModel(m1);

ctrl->simulate();
\end{Verbatim}

\section{Parallel DEVS (Optimistic)}
The simulation of the air traffic on an airport:
\begin{Verbatim}[fontsize=\small]
\\ Setup of the prerequisites
auto tracers = createObject<n_tracers::t_tracerset>();
t_networkptr network = createObject<Network>(2);
std::unordered_map<std::size_t, t_coreptr> coreMap;
std::shared_ptr<Allocator> allocator = createObject<SimpleAllocator>(2);
std::shared_ptr<n_control::LocationTable> locTab = createObject<n_control::LocationTable>(2);

\\ Setup of the cores
t_coreptr c1 = createObject<Multicore>(network, 0, locTab, 2);
t_coreptr c2 = createObject<Multicore>(network, 1, locTab, 2);
coreMap[0] = c1;
coreMap[1] = c2;

\\ Setup of the controller
Controller ctrl("AirControl", coreMap, allocator, locTab, tracers);
ctrl.setPDEVS();
ctrl.setTerminationTime(t_timestamp(720, 0));

\\ Creating and adding the main model
t_coupledmodelptr m = createObject<n_airControl::Airport>("JFK");
ctrl.addModel(m);

ctrl.simulate();
\end{Verbatim}

Here too, the setup is shorter and easier using \textsl{ControllerConfig}:

\begin{Verbatim}[fontsize=\small]
ControllerConfig conf;
conf.name = "AirControl";
conf.simType = Controller::PDEVS;
conf.coreAmount = 2;

auto ctrl = conf.createController();
ctrl->setTerminationTime(t_timestamp(720, 0));

t_coupledmodelptr m1 = createObject<n_airControl::Airport>("JFK");
ctrl->addModel(m1);
ctrl->simulate();
\end{Verbatim}