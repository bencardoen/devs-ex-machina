%TODO conclude the paper, saying that our new simulation tool offers conservative and optimistic synchronization, resulting in good performance, which is even better than adevs
% needs some extending. Maybe clarify that we actually do single core PDEVS and not
% pure classic DEVS.
Both synchronization algorithms offer good performance in differing simulations, in some simulation our kernels outperform adevs whereas in others we can still improve. 
The optimistic implementation needs to be extended with lazy evaluation/cancellation to function in cyclic simulations.
\subsection{Future work}
%TODO future work: more things that we might want to implement, and probably something about automatically switching between synchronization protocols at runtime (and reference my activity papers ;))
\subsubsection{Activity}
As shown in \cite{PythonPDEVS_ACTIMS} activity and allocation of models across kernels is a key aspect in achieving high performance in any parallel implementation. Allocating models so that there are no dependency cycles between their containing kernels is a first step, but not always possible. For optimistic one can use re-allocation to break (runtime dependency) cycles or perform load balancing. If kernels are unevenly balanced they will begin to drift fast, causing increasingly more reverts. 
\subsubsection{Hybrid}
The optimistic implementation could use (null/eot/eit) from conservative to detect and/or reduce the cost of reverts without completely stalling on influencing kernels.
Conservative kernels could be extended with runtime information about influencing models, if one can guarantee a static dependency is not used for a fixed time-span, this dependency can be removed for that period of (virtual) time. \\
Ultimately the simulation could switch at runtime between protocols based on the information provided by activity tracking.
