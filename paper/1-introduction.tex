\textsf{DEVS}~\cite{ClassicDEVS} is a popular formalism for modelling complex dynamic systems using a discrete-event abstraction.
In fact, it can serve as a simulation ``assembly language'' to which models in other formalisms can be mapped~\cite{DEVSbase}.
A number of tools have been constructed by academia and industry that allow the modelling and simulation of \textsf{DEVS} models.

But with the ever increasing complexity of simulation models, parallel simulation becomes necessary to perform the simulation within reasonable time bounds.
And while \textsf{Parallel DEVS}~\cite{ParallelDEVS} was introduced to increase parallelism, this is often insufficient.
Several synchronization protocols from the discrete event simulation community~\cite{FujimotoBook} have been applied to \textsf{DEVS} simulation.
While several parallel \textsf{DEVS} simulation kernels exist, they are often limited to a single synchronization protocol.
The reason for different synchronization protocols, however, is that their distinct nature makes them applicable in different situations, each outperforming the other in specific models.
The applicability of parallel simulation capabilities of current tools is therefore limited.

Users who simulate a wide variety of models, with different ideal synchronization protocols, are therefore out of luck:
either they accept lower performance for some models, or they use different simulation kernels.
Neither is acceptable for complex models: performance can decrease to unacceptable levels, and simulation kernel APIs can diverge significantly.

This paper introduces DEVS-Ex-Machina (``dxex''), our simulation tool which offers multiple synchronization protocols: no synchronization (sequential execution), conservative synchronization, or optimistic synchronization.
The selected synchronization protocol is transparant to the simulated model.
Users should merely determine, at the start of simulation, which protocol they wish to use.
Our tool is based on PythonPDEVS, but implemented in C++11, to increase both performance and portability across different platforms.

We implemented a model that, depending on a single parameter, changes its ideal synchronization protocols.
Dxex, then, is used to compare simulation using exactly the same tool, but with a different synchronization protocol.
As dxes is able to simulate it both ways, users can always opt to use the fastest protocol available.
To verify that our flexibility does not counter performance, we compare to adevs, currently one of the fastest \textsf{DEVS} simulation tools available~\cite{PythonPDEVS,DEVSSurvey}.

Dxex can be found online at \url{https://bitbucket.org/bcardoen/devs-ex-machina}.

The remainder of this paper is organized as follows:
Section~\ref{sec:2-background} introduces the necessary background on synchronization protocols.
Section~\ref{sec:3-features} elaborates on some of our features, and our design that enables our flexibility.
In Section~\ref{sec:4-performance}, we evaluate performance of our tool by comparing its different synchronization protocols, as well as a comparison to adevs.
Related work is discussed in Section~\ref{sec:5-related-work}.
Section~\ref{sec:6-conclusion} concludes the paper and gives future work.
