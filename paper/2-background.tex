This section briefly introduces two distinct synchronization protocols, as used by dxex.
Furthermore, we make note of several new features of C++11.

\subsection{Conservative Synchronization}
The first synchronization protocol that we will introduce, is \textit{conservative synchronization}~\cite{FujimotoBook}.
In conservative synchronization, a node is allowed to progress in simulated time, independent of all other nodes, up to the point where it can guarantee that no causality errors can happen.
When this point in time is reached, the node has to block until it is allowed to progress any further.
In practice, this means that all nodes need to be aware of the current simulation time of all other nodes, and the time it takes an event to propagate (called \textit{lookahead}).
Several algorithms are defined in the literature to implement this behaviour.
An overview is given in~\cite{FujimotoBook}.

Deadlock can occur when a dependency cycle occurs and the amount of exchanged messages is low.
Again, multiple algorithms are defined to handle this situation.
Algorithms range from deadlock avoidance to deadlock recovery.

The main advantage of this approach is that it has a very low overhead in cases of high parallelism between different nodes.
Each node can simulate in parallel, while sporadically notifying other nodes that they can progress even further.
The disadvantage, however, is that the amount of parallelism is explicitly limited by the size of the lookahead.
If a node can influence another (almost) instantaneously, no matter how rarely it occurs, the amount of parallelism is severely reduced.
It is also up to the user to define the size of the lookahead, depending on how the model is known to behave.
Slight changes in model behaviour can cause significant changes to the lookahead, and can therefore also have a significant influence on simulation performance.

\subsection{Optimistic Synchronization}
A completely different synchronization protocol is \textit{optimistic synchronization}~\cite{TimeWarp}.
Whereas conservative synchronization would prevent causality errors at all costs, optimistic synchronization will allow them to happen, but correct them afterwards.
Each node is allowed to progress in simulated time as much as possible, without taking note of the state of any other node.
When an event arrives at a node, which is already further in simulated time, the node will have to roll back its state to right before the event would normally have to be processed.
As the simulation time is now rolled back to before the event is processed, the event can simply be processed as if no causality error ever occured.

Rolling back the simulation time requires the node to store past model states, such that they can be restored later on.
Furthermore, all incoming and outgoing events need to be stored as well.
Incoming events need to be passed to the models again, when the correct simulation time has again been reached, and outgoing events need to be cancelled, as potentially a different series of output events would normally have been generated.
Cancelling events, however, can cause further rollbacks, as the receiving node might also have to roll back its state.
In practice, a single causality error could have significant repercussions on the complete simulation.

Further changes are required when the model performs unrecoverable transactions, such as I/O (\textit{e.g.}, tracing, writing to file, printing output), allocating and deallocating memory, and so on.
For this, lightweight synchronization algorithms are still required to determine the lower bound of all simulation times, through the computation of a \textit{Global Virtual Time} (GVT).

%TODO reword advantages
The advantage of optimistic synchronization is that there is no blocking at all: a node can always progress as fast as possible in simulated time.
Additionally, you do not suffer from a small lookahead if the small lookahead situation rarely occurs.
If an (almost) instantaneous event occurs rarely, performance will only be impacted if it occurs.
The main disadvantage is the unpredictable performance and arbitrary cost of rollbacks.
If rollbacks occur rather frequently, simulation quickly becomes slower than sequential, as the overhead of state saving becomes significant.
Apart from the overhead in CPU time, there is also an overhead in terms of memory, as all intermediate states will also have to be stored up to a point where it can be considered \textit{irreversible}.

While optimistic synchronization does not explicitly depends on the lookahead, simulation performance is still bound by the lookahead implicitly.

\subsection{C++11 Parallelism Features}
%TODO explain what is new in C++11 for parallelism and how it contributes to our cause
C++11 offers a wide range of portable synchronization primitives in the Standard Library, whereas in earlier versions one had to resort to non-portable (C) implementations. More importantly, C++11 is the first version of the standard that actually defines a multi-threaded abstract machine memory model in the language.
Our kernels use a wide range of threading primitives and atomic operations. As an example, eot/eit/nulltime are not exchanged as messages, but as reads/writes to atomic fields shared by all kernels. This avoids the otherwise unavoidable latency penalty by mixing simulation messages with synchronization messages. For an in-depth study, we refer to~\cite{CPE:CPE3007}.
Most modern compilers support the full standard, allowing the kernels to be portable by default on any standard compliant platform. 
