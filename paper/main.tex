\documentclass{scs}
\copyrightnotice{
	SpringSim-TMS/DEVS 2016 April 3-6, Pasadena, CA, USA
	
	\copyright 2016 Society for Modeling \& Simulation International (SCS) 
}

% Load basic packages
\usepackage{balance}		% to better equalize the last page
\usepackage{graphics}		% for EPS, load graphicx instead
\usepackage{times}			% comment if you want LaTeX's default font
\usepackage{url}			% llt: nicely formatted URLs
\usepackage{dblfloatfix}	% allow placement of a page-width figure at top or bottom of page

% llt: Define a global style for URLs, rather that the default one
\makeatletter
\def\url@leostyle{%
  \@ifundefined{selectfont}{\def\UrlFont{\sf}}{\def\UrlFont{\small\bf\ttfamily}}}
\makeatother
\urlstyle{leo}

% To make various LaTeX processors do the right thing with page size.
\def\pprw{8.5in}
\def\pprh{11in}
\special{papersize=\pprw,\pprh}
\setlength{\paperwidth}{\pprw}
\setlength{\paperheight}{\pprh}
\setlength{\pdfpagewidth}{\pprw}
\setlength{\pdfpageheight}{\pprh}

% Make sure hyperref comes last of your loaded packages,
% to give it a fighting chance of not being over-written,
% since its job is to redefine many LaTeX commands.
\usepackage[pdftex]{hyperref}

% create a shortcut to typeset table headings
\newcommand\tabhead[1]{\small\textbf{#1}}

% End of preamble. Here comes the document.
\begin{document}

\title{Performance analysis of a parallel PDEVS simulator handling both conservative and optimistic protocols}

% This is not how it should be, but we have 7 authors with difficult affiliations...
\author{
\begin{tabular}{ccc}
    Ben Cardoen\dag     & Stijn Manhaeve\dag    & Tim Tuijn\dag \\
    \multicolumn{3}{c}{\email{\{firstname.lastname\}@student.uantwerpen.be}} \\
    \\
\end{tabular}\\
\begin{tabular}{cc}
    Yentl Van Tendeloo\dag  & Kurt Vanmechelen\dag \\
    Hans Vangheluwe\dag\ddag & Jan Broeckhove\dag \\
    \multicolumn{2}{c}{\email{\{firstname.lastname\}@uantwerpen.be}} \\
    \\
\end{tabular} \\
\affaddr{\dag \space University of Antwerp, Belgium} \\
\affaddr{\ddag \space McGill University, Canada} \\
}

\maketitle

\begin{abstract}
\input abstract.tex
\end{abstract}

%TODO fill in
%\keywords{
%	Guides; instructions; author's kit; conference publications;
%	keywords should be separated by a semi-colon.
%	\textcolor{red}{Optional section to be included in your final version, but strongly encouraged..}
%}

% ACM Classification Keywords
%\category{I.6.1}{SIMULATION AND MODELING (e.g. Model Development). }
%See: \url{http://www.acm.org/about/class/1998/} for more information and the full list of ACM classifiers and descriptors.
%\textcolor{red}{Optional section to be included in your final version, but strongly encouraged.}

\section{Introduction}
\input 1-introduction.tex

\section{Background}
\input 2-background.tex

\section{Features}
\input 3-features.tex

\section{Performance}
\input 4-performance.tex

\section{Related Work}
\input 5-related-work.tex

\section{Conclusions}
\input 6-conclusions.tex

\section*{ACKNOWLEDGMENTS}
This work was partly funded with a PhD fellowship grant from the Research Foundation - Flanders (FWO).
%TODO add other acknowledgments if necessary?

%TODO actually cite them; just here to know how long the list of references would be
\nocite{*}

\bibliographystyle{scs}
\bibliography{papers}

\end{document}
