\textsf{DEVS}~\cite{ClassicDEVS} is a popular formalism for modelling complex dynamic systems using a discrete-event abstraction.
In fact, it can serve as a simulation ``assembly language'' to which models in other formalisms can be mapped~\cite{DEVSbase}.
A number of tools have been constructed by academia and industry that allow the modelling and simulation of \textsf{DEVS} models.

But with the ever increasing complexity of simulation models, parallel simulation becomes necessary to perform the simulation within reasonable time bounds.
Whereas \textsf{Parallel DEVS}~\cite{ParallelDEVS} was introduced to increase parallelism, its inherent parallelism is often insufficient~\cite{Himmelspach}.
Several synchronization protocols from the discrete event simulation community~\cite{FujimotoBook} have been applied to (\textsf{Parallel}) \textsf{DEVS} simulation~\cite{globaltimewarp}.
With synchronization protocols, different simulation cores can be at different points in simulated time, significantly increasing parallelism, at the cost of synchronization overhead.
While several parallel \textsf{DEVS} simulation kernels exist, they are often limited to a single synchronization protocol.
The reason for different synchronization protocols, however, is that their distinct nature makes them applicable in different situations, each outperforming the other for specific models~\cite{Jafer}.
The parallel simulation capabilities of current tools are therefore limited to specific domains.

This paper introduces \textit{DEVS-Ex-Machina}\footnote{\url{https://bitbucket.org/bcardoen/devs-ex-machina}} (``\textit{dxex}''): our simulation tool~\cite{dxex} which offers multiple synchronization protocols: no synchronization (single core execution), conservative synchronization, or optimistic synchronization. Each inter core synchronization protocol can be complemented with the parallel execution of concurrent events within a simulation core. We refer to this intra core protocol as \pSim for the remainder of this paper. \\
This results in the following list of available parallelization approaches:\\
\begin{itemize}
\item No inter core synchronization (single core), no intra core synchronization : S
\item Conservative inter core synchronization, no intra core synchronization : C
\item Optimistic inter core synchronization, no intra core synchronization : O
\item No inter core synchronization (single core), pSim intra core synchronization : SP
\item Conservative inter core synchronization, pSim intra core synchronization : CP
\item Optimistic inter core synchronization, pSim intra core synchronization : OP
\end{itemize}
The selected synchronization protocol is transparent to the simulated model: users only determine the protocol to use.
Users who simulate a wide variety of models, with different ideal synchronization protocols, can run the exact same model using the exact same tool, but with different synchronization protocols.
As model behaviour, and thus the ideal synchronization protocol, might change during simulation, runtime switching of synchronization protocols is also supported.
This runtime switching can be based on performance metrics, which are logged during simulation.
Information is made available to a separate component, where a choice can be made about which synchronization protocol to use. % It's manual now, can be extended to full automatic.
Additionally, we investigate how model allocation influences the performance of our synchronization protocols.
To this end, we have included an allocation component in our simulation kernel.

Our tool is based on \textit{PythonPDEVS}~\cite{PythonPDEVS}, but implemented in C++11 for increased performance, using features from the new C++14 standard when supported by the compiler.
Unlike \textit{PythonPDEVS}, \textit{dxex} only supports multicore parallelism, thus no distributed simulation.

Using several benchmark models, we demonstrate the factors influencing the performance under a given synchronization protocol.
Additionally, we investigate a model which changes its behaviour (and ideal synchronization protocol) during simulation.
\textit{Dxex}, then, is used to compare simulation using exactly the same tool, but with a varying synchronization protocol.
With \textit{dxex} users can always opt to use the fastest protocol available, and through its modularity, users could even implement their own, or variants of existing ones.
To verify that this modularity does not hamper performance, we compare to \textit{adevs}~\cite{adevs}, another \textsf{Parallel DEVS} simulation tool.

The remainder of this paper is organized as follows:
Section~\textsc{\nameref{sec:2-background}} introduces the necessary background on synchronization protocols.
Section~\textsc{\nameref{sec:3-features}} elaborates on our design that enables the flexibility to switch protocols.
In Section~\textsc{\nameref{sec:4-performance}}, we evaluate the performance of our tool using the different synchronization protocols, and we also compare with \textit{adevs}'s performance.
We continue by introducing runtime switching of synchronization protocols and different options for model allocation in Section~\textsc{\nameref{sec:4b-hotswap}} and Section~\textsc{\nameref{sec:4a-allocation}}, respectively.
Related work is discussed in Section~\textsc{\nameref{sec:5-related-work}}.
Section~\textsc{\nameref{sec:6-conclusion}} concludes the paper and presents ideas for future work.
