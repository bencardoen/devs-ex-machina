\subsection{Conclusion}
We have shown that our contribution is invaluable for high performance simulation: depending on the expected behaviour, modellers can choose the most appropriate synchronization protocol.
We will summarize the key factors in achieving good parallel performance here. % reword aspects syn
\subsubsection{Allocation}
Allocation of atomic models over kernels is critical to obtain a speedup. 
In the case of the Interconnect model it is not possible to allocate models without introducing a cyclic dependency between kernels with severe performance degradation as a result. 
Even when no cycles exist in the topology between kernels a good allocation scheme will minimize the number of connections between kernels, for example the star topology in PHoldTree with depth first allocation or the chain topology in the Queue model. 
Dxex can optionally offer the user a visualization of the behaviour of the kernels under a user supplied allocation scheme to allow for more insight. 
The same instrumentation could in future be used to optimize an existing allocation scheme.
\subsubsection{Uncertainty}
A simulation where future behaviour cannot be predicted is typically not suited for conservative simulation, but we have demonstrated that dxex's conservative kernel can still offer a non trivial speedup in this context. 
Optimistic is not limited by this uncertainty and can offer a good speedup regardless of lookahead. 
Phold and PholdTree demonstrate that a single model can benefit from different synchronization protocols depending on a single parameter.
\subsubsection{Limits}
Dxex's conservative kernel is constrained by the amount of models it controls, as shown in the PHoldTree benchmark. This is especially the case when the minimal lookahead is $\epsilon$ requiring near constant polling of models for lookahead values. 
Optimistic is not constrained by the number of models it manages, but can quickly consume all available memory in a simulation. While this effect can be reduced with a thread aware allocation scheme, the underlying problem remains.
