\subsection{Conclusions on Performance Evaluation}
We have shown that our contribution is invaluable for high performance simulation: depending on the expected behaviour, modellers can choose the most appropriate synchronization protocol.
PDEVS is compared with the synchronization protocols either in isolation or complementing each other. We have shown that both approaches are useful in specific model configurations.
But even with the right synchronization protocol, we have seen that two problems remain.

First, although one of both synchronization protocols might be ideally suited for specific model behaviour, nothing guarantees that the model will exhibit the same behaviour throughout the simulation.
Similarly to the polymorphic scheduler~\cite{MasterThesis}, we wish to make it possible for the ideal option to be switched during simulation.
When changes to the model behaviour are noticed, the used synchronization protocol can be switched as well. 

Second, the allocation of models is nontrivial and has a significant impact on performance.
While our parallel speedup for the Queue model, for example, was rather high, this is mostly due to characteristics of the model: the dependency graph does not contain any cycles.
When cycles were introduced, as in the Interconnect model, performance became disastrous.

In the next two sections, we elaborate on these two problems.
